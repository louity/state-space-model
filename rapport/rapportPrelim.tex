\documentclass[11pt, oneside]{amsart}
\usepackage{geometry}
\geometry{letterpaper}
\usepackage[francais]{babel}
\usepackage[utf8]{inputenc}
\usepackage{graphicx}
\usepackage{float}
\usepackage{etex,mathtools}
\usepackage{amssymb}
\usepackage{enumitem}
\usepackage{amsmath}
\usepackage{caption}
\usepackage{listings}
\usepackage{array}

\graphicspath{{images/}}

\title[]{Preliminary report, PGM}
\author[1]{Louis Thiry}
\author[2]{Thomas Kerdreux}
\author[3]{Nicolas Jouvin}
\begin{document}

\maketitle

\section{Understanding of the paper}

This paper presents an algorithm to learn non linear stationnary dynamics of a system.

The system can be viewed as a graphical model :
% include SVG of graphical model

The dynamics of the system can be written in general as :
\begin{align*}
  % equation non linéraire
\end{align*}

The first key assumption made to simplify the problem is :
\begin{align*}
  % dev taylor
\end{align*}
% justification, commentaire

The second key assumption made to simplify the problem is :
\begin{align*}
  % f et g affines + somme RBF
\end{align*}
% justification, commentaire

The algorithm is tested first on generated data.
% commentaire sur la forme bien particulière de ces fonction : douille

\section{Planned implementations}

1. Kalman filter

\end{document}
