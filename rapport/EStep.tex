\section{E-Step}

At iteration $k$, in the E-step, given the parameter $\theta^{(k-1)} = (\theta_f^{(k-1)}, \theta_g^{(k-1)})$ computed in the iteration $k-1$, we want to infer the sequences $(x_t)_{t=1 \ldots T}$ and $(x_t, x_{t+1})_{t=1 \ldots T-1}$ knowing the output sequence $(y_t)_{t=1 \cdots T}$ and the input sequence $(u_t)_{t=1 \cdots T}$.
In other words we want to compute the conditional probabilities :
\begin{align*}
  &p_{\theta^{(k-1)}}\left(x_t|y_1, u_1, \ldots, y_T, u_T \right ), t=1 \ldots T\\
  &p_{\theta^{(k-1)}}\left(x_t, x_{t+1}|y_1, u_1, \ldots, y_T, u_T \right ), t=1 \ldots T-1\\
\end{align*}

If we use the dynamics equations with $f$ and $g$, we have no guarantee that given a Gaussian probability for $x_t$, we have a Gaussian probability for $x_{t+1}$ and the computations become intractable.
To solve this problem, the authors propose to linearize the dynamics equations as each time step $t$:
\begin{align*}
	x_{t+1} &= f(\hat{x}_t, u_t) + A_{\hat{x}_t} (x - x_t) + w_t\\
  A_t &= \frac{\partial f}{\partial x}(\hat{x}_t)\\
	y_t &= g(\hat{x}_t, u_t) + C_{\hat{x}_t} (x - x_t) + v_t\\
  C_t &= \frac{\partial g}{\partial x}(\hat{x}_t)\\
\end{align*}
where $\hat{x}_t$ is a well chosen point.
Then the functions are locally linear, thus all probabilities are Gaussian and we can adapt the classical inferences algorithms (\textbf{Kalman filter} and \textbf{Rauch-Tung-Stribel smoother}).

This approach involving linearization at each time step is called Extented Kalman Smoothing (EKS).

\subsection{Kalman filter}

If the case where $f$ and $g$ are affine ($I=0$ et $J=0$) :
\begin{align*}
  f(x,u) &= Ax + Bu + b\\
  g(x,u) &= Cx + Du + d\\
\end{align*}
the Kalman filter algorithm computes dynamically the conditional probabilities :
\begin{align*}
  &p_{\theta^{(k-1)}}\left(x_t|y_1, u_1, \ldots, y_t, u_t \right ), t=1 \ldots T\\
  &p_{\theta^{(k-1)}}\left(x_t, x_{t+1}|y_1, u_1, \ldots, y_t, u_t \right ), t=1 \ldots T\\
\end{align*}
Since the probability of the first state $x_0$ and the noise sequences $(w_t)$ and $(v_t)$ are Gaussian and $f$ and $g$ are affine, all the probabilities are Gaussian.
Thus we need to compute means and covariances.

The authors derive the computations of the Kalman filter only in the case where $B,b,D$ and $d$ are all zero.
They don't give any reason for these simplifications.
Is it lazyness?
Is it intended for pedagoical reason?

We adopt the following notations:
\begin{align*}
  \hat{x}_{t|t} &= \mathbb{E}_p \left (x_t|y_1, \ldots , y_t, u_1, \ldots , u_t \right ) \\
  \hat{x}_{t+1|t} &= \mathbb{E}_p \left (x_{t+1}|y_1, \ldots , y_t, u_1, \ldots , u_t \right ) \\
  \hat{P}_{t|t} &= \mathbb{E}_p \left ((x_t - \hat{x}_{t|t})(x_t - \hat{x}_{t|t})^T|y_1, \ldots , y_t, u_1, \ldots , u_t \right ) \\
  \hat{P}_{t+1|t} &= \mathbb{E}_p \left ((x_{t+1} - \hat{x}_{t+1|t})(x_{t+1} - \hat{x}_{t+1|t})^T|y_1, \ldots , y_t, u_1, \ldots , u_t \right ) \\
  \hat{P}_{t,t+1|t} &= \mathbb{E}_p \left ((x_t - \hat{x}_{t|t})(x_{t+1} - \hat{x}_{t+1|t})^T|y_1, \ldots , y_t, u_1, \ldots , u_t \right ) \\
\end{align*}
With these quanities, we can compute the conditionnal probabilities of interest.
The Kalman filter gives us a recursion equation to compute these quanities :
\begin{itemize}
  \item \textbf{Initialization}\\
    \begin{align*}
      \hat{x}_{1|0} &= 0\\
      \hat{P}_{1|0} &= \Sigma_0\\
    \end{align*}
  \item \textbf{Forward recursion}\\
    For $t=1 \ldots T$ :
    \begin{align*}
      \hat{x}_{t+1|t} &= A \hat{x}_{t|t} + B u_t + b\\
      \hat{P}_{t+1|t} &= A \hat{P}_{t|t} A^T\\
      \hat{P}_{t,t+1|t} &= \hat{P}_{t|t}A^T\\
      K_{t+1} &= \hat{P}_{t+1|t}C^T(C \hat{P}_{t+1|t} C^T + R)^{-1}\\
      \hat{x}_{t+1|t+1} &= x_{t+1|t} + K_{t+1} \left (y_{t+1} - (C\hat{x}_{t+1|t} + D u_t + d) \right )\\
      \hat{P}_{t+1|t+1} &= \hat{P}_{t+1|t} - K_{t+1} C \hat{P}_{t+1|t}\\
    \end{align*}
    We compute the matrix $K_{t}$ (Kalman gain matrix) to simplify the computations.
\end{itemize}

In our case, $f$ and $g$ are not linear.
So for $t=1..T$ we linearize our functions $f$ and $g$ around the point $\tilde{x}_t$ and then we can apply the recursion equation below.
We define $\tilde{x}_t$ as following :
\begin{align*}
  \tilde{x}_1 &= \hat{x}_{1|0}\\
  \tilde{x}_{t+1} &= f(\hat{x}_{t|t}, u_t)\\
\end{align*}
Then, in the recursion formula, we just have to replace the matrices $A$ and $C$  by the matrices $A_t$ and $C_t$ :
\begin{align*}
  A_t &= \frac{\partial f}{\partial x}(\tilde{x}_t, u_{t-1})\\
  C_t &= \frac{\partial g}{\partial x}(\tilde{x}_t, u_t)\\
\end{align*}
that we can compute in term of our parameter $\theta = \left( h_1, \ldots , h_I, A, B, b, k_1, \ldots , k_J, C, D, d \right)$ and of the centers and width of our RBF functions $(c_1, S_1, \ldots ,c_I, S_I, c'_1, S'_1, \ldots , c'_J, S'_J)$:
\begin{align*}
  A_t &= A + \sum_{i=1}^I \rho_i(\tilde{x}_t) h_i (\tilde{x}_t - c_i)^T S_i^{-1}\\
  C_t &= \sum_{j=1}^J \rho'_j(\tilde{x}_t) k_j (\tilde{x}_t - c'_i)^T {S'_j}^{-1}\\
\end{align*}
and we replace the vectors $b$ and $d$ by the vectors $b_t$ and $d_t$ :
\begin{align*}
  b_t &= b + f(\tilde{x}_t, u_t)\\
  d_t &= d + g(\tilde{x}_t, u_t)\\
\end{align*}
and we get the conditionnal probabilities of interest :
\begin{align*}
  &p_{\theta^{(k-1)}}\left(x_t|y_1, u_1, \ldots, y_t, u_t \right ), t=1 \ldots T\\
  &p_{\theta^{(k-1)}}\left(x_t, x_{t+1}|y_1, u_1, \ldots, y_t, u_t \right ), t=1 \ldots T-1\\
\end{align*}

\subsection{Rauch-Tung-Stribel smoother}
The authors provide no computations for the RTS.
So all the computations done here were inspired from the computations done in the lecture notes.

If the case where $f$ and $g$ are affine, the RTS smoother algorithm uses the conditionnal probabilities computed with Kalman filter to compute the conditional probabilities :
\begin{align*}
  &p_{\theta^{(k-1)}}\left(x_t|y_1, u_1, \ldots, y_T, u_T \right ), t=1 \ldots T\\
  &p_{\theta^{(k-1)}}\left(x_t, x_{t+1}|y_1, u_1, \ldots, y_T, u_T \right ), t=1 \ldots T-1\\
\end{align*}
Once again, theses probabilities are Gaussian and only we need to compute their means and covariances.

We adopt the following notations:
\begin{align*}
  \hat{x}_{t|T} &= \mathbb{E}_p(x_t|y_1, \ldots , y_T, u_1, \ldots , u_T) \\
  \hat{P}_{t|T} &= \mathbb{E}_p \left ((x_t - \hat{x}_{t|T})(x_t - \hat{x}_{t|T})^T|y_1, \ldots , y_T, u_1, \ldots , u_T \right ) \\
  \hat{P}_{t,t+1|T} &= \mathbb{E}_p \left ((x_t - \hat{x}_{t|T})(x_{t+1} - \hat{x}_{t+1|T})^T|y_1, \ldots , y_T, u_1, \ldots , u_T \right ) \\
\end{align*}
With these quanities, we can compute the conditionnal probabilities of interest.
The Kalman filter gives us a recursion equation to compute these quanities :
\begin{itemize}
  \item \textbf{Initialization}\\
    For the initialization, we need run a Kalman filter to get the quantities $\hat{x}_{t|t}, \hat{P}_{t|t}$ for $t=1 \ldots T$ and $\hat{x}_{t+1|t}, \hat{P}_{t+1|t}, \hat{P}_{t,t+1|t}$ for $t=a \ldots T-1$.
  \item \textbf{Backward recursion}\\
    For $t=T-1 \ldots 1$ :
    \begin{align*}
      L_t &= \hat{P}_{t|t}A^T\hat{P}_{t+1|t}^{-1}\\
      \hat{x}_{t|T} &= \hat{x}_{t|t} + L_t(\hat{x}_{t+1|T} - \hat{x}_{t+1|t})\\
      \hat{P}_{t|T} &= \hat{P}_{t|t} + L_t(\hat{P}_{t+1|T} - \hat{P}_{t+1|t})L_t^T\\
      \hat{P}_{t,t+1|T} &= \hat{P}_{t,t+1|t} - (\hat{x}_{t|t} - \hat{x}_{t|T})(\hat{x}_{t+1|t} - \hat{x}_{t+1|T})^T\\
    \end{align*}
    We compute the matrix $L_{t}$ to simplify the computations.
\end{itemize}

In our case, $f$ and $g$ are not linear.
We linearize our around the point $\dot{x}_t$ and then we can apply the recursion equation below.
We define  as follow :
\begin{align*}
  \dot{x}_t &= \hat{x}_{t|t} = \mathbb{E}_p(x_t|y_1, \ldots , y_t, u_1, \ldots , u_t) \\
\end{align*}
As we see, $\dot{x}_t$ can be seen as the mean of probability  $p_{\theta^{(k-1)}}\left(x_t|y_1, u_1, \ldots, y_t, u_T \right )$ computed by the Kalman filter.
We also see that in the recursion forumla, the matrix $C$ and the vectors $b$ and $d$ do not appear.
So we just have to replace the matrix $A$ by the matrix $A_t$ :
\begin{align*}
  A_t &= \frac{\partial f}{\partial x}(\dot{x}_t, u_{t-1})\\
\end{align*}
The computation of $A_t$ in term of our parameter $\theta$ and of the centers and width of our RBF functions is the same as in the Kalman filter, replacing $\tilde{x}$ by $\dot{x}$.

So we end up with the means and covariance matrices of the probabilities
\begin{align*}
  &p_{\theta^{(k-1)}}\left(x_t|y_1, u_1, \ldots, y_T, u_T \right ), t=1 \ldots T\\
  &p_{\theta^{(k-1)}}\left(x_t, x_{t+1}|y_1, u_1, \ldots, y_T, u_T \right ), t=1 \ldots T-1\\
\end{align*}
which are extactly what we wanted to compute during the E-step.
So we can move to the M-step.
