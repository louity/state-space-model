\documentclass[11pt, oneside]{amsart}
\usepackage{geometry}
\usepackage[utf8]{inputenc}
\usepackage{graphicx}
\usepackage{float}
\usepackage{etex,mathtools}
\usepackage{amssymb}
\usepackage{enumitem}
\usepackage{amsmath}
\usepackage{caption}
\usepackage{listings}
\usepackage{array}

\usepackage[colorinlistoftodos]{todonotes}

\newcommand{\notes}[1]{\ifnotesw \textcolor{red}{  $\clubsuit$\ {\sf \bf \it  #1}\ $\clubsuit$ }\fi}

\title[]{Article review concerning learning}
\author[1]{Nicolas Jouvin}
\author[2]{Thomas Kerdreux}
\author[3]{Louis Thiry}
\begin{document}

\maketitle
\begin{abstract}
  Review of articles concerning learning
\end{abstract}

\begin{itemize}
  \item \textbf{NONLINEAR DYNAMICAL SYSTEM IDENTIFICATION FROM UNCERTAIN AND INDIRECT MEASUREMENTs, HENNING U. VOSS and JENS TIMMER, JURGEN KURTHS}: not learning, article cited as an extension
  \item \textbf{System identification of non-linear state-space models}: E-Step with particle filter, M-Step with gradient descent, no parametrisation explicited.
  \item \textbf{Identification of nonlinear systems using Polynomial Nonlinear State Space models} : Polynomial functions, no EM algo.
  \item \textbf{An Unscented Kalman Filter Approachto the Estimation of NonlinearDynamical Systems Models}: Only E-step with UKF.
  \item \textbf{Dynamical modeling with kernels for nonlinear time series prediction}: Only state varialbes and deals with modelling.
  \item \textbf{Approximate EM Algorithms for Parameter and State Estimation inNonlinear Stochastic Models}: use EKF in E-Step, iterative optimization in M-Step. Cite ou article and its previous version. No further going.
  \item \textbf{State-Space Inference and Learning with Gaussian Processes}: use Gaussian process in E-Step, iterative optimization in M-Step. Cite our article with the term RBF
  \item \textbf{Time series filtering, smoothing and learning using the kernel Kalman filter}: EM algoritmh applied to $\Phi(x)$ with KF in E-Step and gradient descent in M-Step.
  \item \textbf{Dynamic Factor Graphsfor Time Series Modeling}: E end M-Step with gradient.
\end{itemize}



\end{document}
